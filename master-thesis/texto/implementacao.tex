\chapter{Implementa��o} 
\label{capitulo:implementacao}

-- falar que a implementa��o est� baseada em diversos problemas-chave da disserta��o, como o gerenciamento de mem�ria, relevo que n�o pode ser previsto, montanhas que n�o � poss�vel ver por causa do view frustum, etc.

\section{Terreno infinito}

-- falar que utilizarmos o conceito de view frustum (daquele artigo frances) como base para os trabalhos. Em suma, vou descrever como que a coisa toda funciona: o usu�rio s� enxerga o que est� dentro do campo de vis�o, falar da janela de visualiza��o do mundo, etc


\section{Continentes}

-- falar que utilizamos o  algoritmo maluco do professor dos EUA para fazer a costa. Falar que existe o mundo � gigante o suficiente para fazer com que uma matrix que descreve �gua/terra seria invi�vel. Para solucionar esse problema, falar que utilizamos o conceito de isLang local e isLang global. O isLand global d� uma dica se o lugar � �gua ou terra, e o isLand local utiliza multifractais para fazer o desenho das bordas dos continentes.

-- � importante frisar que essa se��o � onde est� a nossa contribui��o na pesquisa: mundo pseudo-infinito, com relevo gerado on-the-fly e com costas de continentes com multiresolu��o.


\section{Relevo}

-- O relevo � gerado on-the-fly por fun��es matem�ticas (explicar essas fun��es); falar que isso � totalmente customiz�vel e que a qualquer momento o relevo pode ser mostrado.